\documentclass[11pt, titlepage]{article}
\usepackage[T1]{fontenc}
\usepackage[utf8]{inputenc}
\usepackage{graphicx}
\usepackage[polish]{babel}
\usepackage{hyperref}
\usepackage{url}
\usepackage{float}

\title{Dokumentacja Projekt 1\\n-body problem MPI}
\date{\today}
\author{<unknown>}

\begin{document}
\maketitle

\section{Wstęp}
\hspace{11pt} Projekt został wykonany w ramach zajęć z Systemów równoległych i rozproszonych. Napisany został w języku C, wykorzystując bibliotekę MPI oraz środowisko MPE do profilowania oraz wizualizacji działania.
\subsection{Opis problemu}
\hspace{11pt} Problem oddziaływania wielu ciał polega na wyznaczaniu ruchu pojedynczych z grupy ciał oddziałujących ze sobą grawitacyjnie.
Ruch ciała wyraża się wzorem:
\begin{equation}
m_i \cdot \frac{\partial^2 \vec{r_i}}{\partial t^2} = \sum^n_{j=1, j\neq i} \frac{G m_i m_j}{\|\vec{r_j} - \vec{r_i}\|} \cdot \frac{\vec{r_j} - \vec{r_i}}{\|\vec{r_j} - \vec{r_i}\|}
\end{equation}
\subsection{Wykorzystany algorytm}
\hspace{11pt} W celu rozwiązania problemu wykorzystany został algorytm Barnesa-Huta. Jest to algorytm aproksymacyjny o złożoności obliczeniowej $O(n log(n))$, zamiast $O(n^2)$, która występuje w przypadku algorytmu prostego sumowania.

Algorytm ten polega na zbudowaniu drzewa, w którego liściach znajduje się maksymalnie jeden obiekt:
\begin{figure}[H]
\center
\caption{Przykładowy podział zbioru danych w drzewie Barnesa-Huta Źródło: \url{https://en.wikipedia.org/wiki/Barnes-Hut_simulation}}
\includegraphics[scale=0.5]{s.png}
\end{figure}

Przestrzeń obliczeniowa jest w każdej iteracji dzielona na cztery części tak długo, aż w każdej z podzielonych części znajdzie się maksymalnie jedno ciało. Lokalnie, czyli między najbliższymi obiektami, siła liczona jest bezpośrednio. Jeżeli obiekty są odległe, czyli odległość środka masy części, której oddziaływanie jest liczone, jest 10 razy większa niż szerokość wydzielonej części, w której znajduje się obiekt brany pod uwagę jest ten środek masy, a nie indywidualne obiekty wewnątrz tej części. 

\section{Struktura Projektu}
\hspace{11pt} Projekt składa się z pięciu plików źródłowych napisanych w języku C, oraz trzech plików nagłówkowych:
\begin{itemize}
\item data.h: w pliku zdefiniowane są struktury danych Vector2D oraz Data, które przechowują dane dotyczące cząstek, które ze sobą oddziałują.
\item drawer.c, drawer.h: pliki służące do rysowania cząstek, wykorzystując bibliotekę MPE.
\item main.c: plik zawierający punkt startowy programu. Przeprowadza parsowanie argumentów, pliku źródłowego, tworzy typ środowiska MPI oraz przeprowadza symulację.
\item parser.c: plik, w którym znajduje się definicja funkcji parsującej plik z danymi wejściowymi.
\item profile.c: plik, który zawiera definicje funkcji służących do przeprowadzenia profilowania. Aby profilowanie było możliwe należy kompilować z flagą \textbf{$-DMPI\_PROFILE$}
\item tree.c, tree.h: piki zawierające deklaracje oraz definicje funkcji wykorzystywanych w obsłudze drzewa Barnesa-Huta.
\end{itemize}

Do projektu dołączone zostały również dwa pliki napisane w języku python, służące do generowania danych:
\begin{itemize}
\item generateData.py: generuje w losowych punktach cząstki z losowymi masami i prędkościami. Dane zapisywane są w pliku o nazwie \textbf{dataFile}
\item gensystem.py: generuje układ planetarny składający się z gwiazdy oraz siedmiu planet. Dane są zapisywane w pliku o nazwie \textbf{planets.data}
\end{itemize}
\section{Implementacja}
\section{Kompilacja}
\section{Działanie}
\section{Podsumowanie}
\end{document}